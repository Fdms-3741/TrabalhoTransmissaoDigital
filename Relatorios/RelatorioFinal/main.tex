\documentclass{article}
\usepackage{graphicx} % Required for inserting images

\usepackage[T1]{fontenc}
\usepackage[utf8]{inputenc} 
\usepackage[brazilian]{babel}

\usepackage[a4paper,total={6in,8in}]{geometry}

\usepackage{amsmath}
\usepackage{amssymb}

\usepackage[linesnumbered,ruled,portuguese]{algorithm2e}
\SetKwComment{Comment}{/* }{ */}
\SetEndCharOfAlgoLine{}
\SetKwInput{KwData}{Recebe}
\SetKwInput{KwResult}{Retorna}

\usepackage[style=numeric,sorting=none]{biblatex}
\addbibresource{TransmissaoDigital.bib}
\usepackage{csquotes}

% Usa as macros para vetores e matrizes
\usepackage{MathMacros}

\title{Relatório Final de Transmissão Digital}
\author{Fernando Dias de Mello Silva}
\date{6 de Dezembro 2023}

\begin{document}

\maketitle

\section{Introdução}

Esse relatório tem como objetivo reproduzir alguns dos resultados do artigo de~\cite{li.etal_2017a}. O trabalho de~\cite{li.etal_2017a} consiste em avaliar o desempenho de estimadores de canal para o caso de sistemas \textit{massive} MIMO que fazem uso de quantizadores de 1 bit de resolução. A estimação de canal é feita com o uso da decomposição de Bussgang para obter um modelo linear estatísticamente equivalente do processo não linear de quantização. 

Esse relatório descreve o processo de reprodução de alguns resultados do artigo. Para isso, desenvolveu-se um ambiente de simulação que contém todas as operações de transmissão de um sistema MIMO. Esse sistema serviu para fazer simulações de taxas de erros de bit, simulações de cálculo de capacidade e estimação de canal. Com isso, além de validar alguns resultados do artigo também foi possível explorar na prática alguns resultados teóricos já esperados para a comunicação MIMO.

As seções desse relatório são separadas por experimentos, com fundamentos, metodologia e resultados autocontidos. Com isso, a divisão deste relatório é feita da seguinte forma:
Na seção~\ref{sec:ambiente_trabalho}, será apresentado o ambiente de simulação utilizado e as principais ferramentas, assim como formas de executar o código da simulação.
Na seção~\ref{sec:fundamentos_mimo}, são descritos os algoritmos e operações fundamentais para o processo de transmissão MIMO, com funções que serão utilizadas no resto do relatório. 
Na seção~\ref{sec:estimacao_canal}, serão descritos os algoritmos de estimação de canal utilizados e o experimento de comparação de erro de estimação entre eles.
Na seção~\ref{sec:capacidade_canal}, será descrito quais as operações de capacidade de canal utilizadas e os experimentos relacionados a eficiência espectral alcançada.
Na seção~\ref{sec:conclusao}, serão feitas as considerações finais sobre o trabalho.

Ao longo do trabalho, $\Cset$ é o conjunto dos números complexos e $\Rset$ o dos números reais. Vetores estão em negrito e são declarados como em $\xbf\in\Cset^{N}$. Matrizes, com $M$ linhas e $N$ colunas, estão em letras maiúsculas e são representadas como em $A\in\Rset^{M\times N}$. A função $\Vect{A}$ é a função de vetorização da matriz $A$. O operador $\otimes$ é o produto de Kroneker. A função $\Diag{\abf}$ é a função de diagonalização de um vetor $\abf$.  Quando um vetor ou uma matriz é aleatória, a notação $\sim$ indica a sua distribuição respectiva. $\Ncal(\mu,\sigma)$ é a distribuição normal no conjunto dos reais com média $\mu$ e desvio padrão $\sigma$, e a distribuição $\Ccal\Ncal(\mu,\sigma)$ é a distribuição normal complexa. 


\section{Ambiente de trabalho}
\label{sec:ambiente_trabalho}

A simulação foi feita com o uso da linguagem de programação \texttt{Python 3}. As principais bibliotecas utilizadas são as bibliotecas \texttt{NumPy}, \texttt{CuPy} e \texttt{SciPy} para cálculos numéricos e funções executadas em GPU, a biblioteca \texttt{Pandas} para manipulação de tabelas de resultados, a biblioteca \texttt{cvxpy} para otimização convexa e as bibliotecas \texttt{Matplotlib} e \texttt{Seaborn} para visualização dos resultados. Funções e métodos de bibliotecas estarão acompanhados pelo nome da respectiva biblioteca. Quando este não for o caso, a função citada está no código desse relatório. 

O ambiente de trabalho utilizado foi o \texttt{Jupyter}, esse ambiente permite a programação de ``\textit{notebooks}'' formados por uma estrutura de texto HTML e código que pode ser executado e interagido diretamente. A google oferece uma plataforma chamada Collab para a execução desses \textit{notebooks} em núvem. O código respectivo desse relatório está em um notebook do Collab que pode ser acessado através do link do repositório desse trabalho.

Os códigos, relatórios e outros materiais relacionados estão na página do github\footnote{https://github.com/Fdms-3741/TrabalhoTransmissaoDigital}.

\section{Fundamentos MIMO}
\label{sec:fundamentos_mimo}

Para dar início a investigação do trabalho de~\cite{li.etal_2017a}, deve-se primeiramente criar o ambiente para comunicações MIMO e validá-lo com simulações. Portanto, nessa seção será investigado os fundamentos de comunicação MIMO e será proposto simulações para validação e exploração de alguns conceitos teóricos.  

\subsection{Fundamentação teórica}

Como visto no relatório anterior, temos que a transmissão MIMO de $\tau$ símbolos entre $K$ transmissores e $M$ receptores consiste na seguinte equação:
\begin{equation}
    Y = \sqrt\rho HS + W
    \label{eq:transmissao_mimo}
\end{equation}

Em que $Y\in\Cset^{M\times\tau}$ são os símbolos recebidos pelos $M$ receptores, $S\in\Cset^{K\times\tau}$ são os símbolos transmitidos pelos $K$ transmissores, $W\in\Cset^{M\times\tau}\sim\Ccal\Ncal(0,1)$ é o ruído AWGN e $H\in\Cset^{M\times K}\sim\Ccal\Ncal(0,1)$ é a matriz de canal. Devido a diversidade espacial desse tipo de comunicação, pode-se considerar que a matriz $H$ é do tipo Rayleigh e sem memória. 

Para a recepção, são utilizados dois receptores da literatura: O \textit{Maximum Ratio Combining} (MRC) e o \textit{Zero Forcing} (ZF). Ambos os receptores estão descritos abaixo: 

\begin{equation}
\begin{cases}
    W^T_{ZF} = H^H \\
    W^T_{MRC} = (H^HH)^{-1}H^H
\end{cases}
\end{equation}

Diversos trabalhos da literatura~\cite{} exploram o desempenho de ambos receptores, e é observado que o desempenho do XXX é melhor do que o do YYY. 

Já um resultado fundamental da diversidade de recepção, demonstrado em~\cite[eq. 3.70 e 3.71]{tse.viswanath_2005}, é que com o aumento do número de antenas, a taxa de erro de transmissão diminui devido tando a um ganho de potência quanto um ganho de diversidade. 

\subsection{Descrição dos experimentos}

Para demonstrar o processo de transmissão e observar os fenômenos citados na subseção anterior, são feitas simulações de Monte Carlo do processo de transmissão para múltiplos valores de SNR e de número de antenas. O experimento de transmissão MIMO é descrito pelo algoritmo~\ref{alg:ber_snr}. O experimento consiste em uma simulação de Monte Carlo do processo de transmissão. Cada iteração do experimento consiste em gerar um novo canal $H$ e as matrizes de recepção para o ZF ($W^T_{ZF}$) e o MRC ($W^T_{MRC}$). Após isso, para cada valor de SNR e cada valor de quantidade de antenas receptoras, é gerado uma matriz de bits $B$ que é codificada, transmitida, recebida (para cada receptor) e decodificada. Esse processo resulta na matriz de bits estimada $\hat{B}$, que pode ser comparada com a matriz $B$ para o cálculo do BER dessa transmissão. 

As funções \texttt{CodificarQPSK(MatrizBits)} e \texttt{DecodificarQPSK(MatrizSimbolos)} é responsável pela conversão entre matrizes de bits e matrizes de símbolos QPSK normalizados. A modulação consiste no mapeamento de dois bits ($\{11,10,01,00\}$) para um símbolo QPSK ($\{1+j,1-j,-1+j,-1-j\}1/\sqrt{2}$). Já a demodulação consiste na aplicação de um critério de decisão para o mapeamento de um número complexo $s$ qualquer para um número $s_{QPSK}$ contido na constelação antes de aplicar a operação inversa da modulação. Esse critério de decisão é definido como $s_{QPSK}=sign(Re(s))+sign(Im(s))$, onde $sign()$ é a função sinal. A função \texttt{Quantizar($S$)}, usada nas próximas seções, apenas aplica o mesmo critério de decisão da função \texttt{DecodificarQPSK()}, sem a conversão de símbolos para bits. 

Para realizar a transmissão MIMO, foi desenvolvida a função \texttt{TransmitirMIMO}, representada pelo algoritmo~\ref{alg:transmissao_mimo}. Ela recebe o canal, a matriz de sinal e a relação sinal ruído da transmissão. O ruído é gerado pela função descrita na linha~\ref{line:ruido} que gera uma matriz de dimensões $A\times B$ em que cada elemento é uma amostra de uma distribuição normal com média $\mu$ e desvio padrão $\sigma$. Nessa função, é possível ver que o ruído é normalizado ($\sigma=1$) para que a relação sinal ruído seja dada pelo ganho de transmissão $\rho$, que é calculado em função do SNR. O valor de $\rho$ também é dividido entre as antenas transmissoras, assim a potência de transmissão é dividida entre todas as antenas e a relação SNR se torna global. 

\begin{algorithm}
    \label{alg:transmissao_mimo}
    \caption{Função \texttt{TransmitirMIMO}}
    \KwData{$H,\sbf,SNR$}
    \KwResult{$\ybf$}
    \Comment{Exemplo de comentário. Variáveis e operadores serão representados por notação matemática convencional.}
    $\wbf \gets $\texttt{numpy.random.uniform(0,1,size=(M,$\tau$))}\; \label{line:ruido}
    $\rho \gets (10^{\frac{SNR}{10}})/K$\;
    $\ybf \gets \sqrt{\rho}H\sbf+\wbf$\;
\end{algorithm}

\subsection{Resultados}


Ao total, foram feitos dois conjuntos de experimentos: Um com M fixo e SNR variável e outro com M variável e SNR fixo. Para ambos, foram feitas $10000$ repetições em que $K=4$ e $\tau=10^7$. Para o $M=16$ fixo, o SNR varia entre -20 e 20 com passo 5 e para o $SNR = 0$ dB os valores de M eram $M=\{4,8,16,32,64\}$. Os resultados podem ser vistos nas figuras~\ref{fig:resultado_ber_snr} e~\ref{fig:resultado_ber_m}. 

Para a figura~\ref{fig:resultado_ber_snr}, demonstrou-se a diferença de desempenho entre os receptores, que aumenta quanto maior for o SNR. A figura~\ref{fig:resultado_ber_m} também mostra essa difernça e destaca que, na condição de maior quantidade de antenas receptoras, a diferença de desempenho aumenta ainda mais entre os receptores.

\begin{algorithm}
    \label{alg:ber_snr}
    \caption{Estimativa do BER via MonteCarlo}
    \KwData{$k,\tau,\text{repetições},\text{ListaM},\text{ListaSNR}$}
    \KwResult{$resultados$}
    $resultados = []$\Comment*[l]{Lista vazia para receber resultados}
    \Comment{Início das simulações}
    \For{(i=0;$i<\text{repetições};i=i+1$)}{ \label{line:ber_snr_rep_sim}
        \For{$M$ \textbf{em} ListaM}{
            \Comment{Canal novo gerado por iteração}
            $H \gets$ \texttt{GerarCanal($m$,$k$)}\;
            \Comment{Receptores calculados a partir do canal real}
            $W^T_{MRC} \gets$ \texttt{CalcularReceptorMRC(H)}\;
            $W^T_{ZF} \gets$ \texttt{CalcularReceptorMRC(H)}\;
            \For{$SNR$ \textbf{em} ListaSNR}{
                \Comment{Bits novos gerado por transmissão}
                $B \gets$ \texttt{GerarBits($k$,$\tau$)}\;
                $S \gets$ \texttt{CodificarQPSK($B$)}\;
                $Y \gets$ \texttt{TransmissaoMIMO(H,S,SNR)}\;
                \For{$receptor$ \textbf{em} $Receptores$}{
                    $\hat{S} \gets W^T_{receptor}Y$\;
                    $\hat{B} \gets$ \texttt{DecodificarQPSK($S$)}\;
                    $BER \gets$ \texttt{SomarElementos($|\hat{B}-B|$)}$/(k.\tau)$\;
                    \Comment{Adiciona o resultado dessa iteração na lista de resultados}
                    $resultados$\texttt{.concatenar($[i,SNR,receptor,BER]$)}\;
                }
            }
        }
    }
\end{algorithm}


\section{Estimação de canal}
\label{sec:estimacao_canal}

Já nessa seção, o objetivo será observar o desempenho de estimação de canal para diferentes estimadores, tanto o proposto pelo artigo como outras estimações, que variam entre outras propostas e a estimação convencional via mínimos quadrados. 

\subsection{Fundamentação teórica}

A estimação de canal é feita com a transmissão de símbolos piloto, representados pela matriz $\Phi\in\Cset^{\tau\times K}$, e a equação de transmissão é a mesma vista na eq.~\eqref{eq:transmissao_mimo}. Após a transmissão, os símbolos são quantizados com base na relação $R_p=\mathcal{Q}(Y_p)$, que quantiza cada elemento da matriz. Finalmente, o canal pode ser estimado com base no seguinte sistema:

\begin{equation}
    \label{eq:min_squa}
    \rbf_p = \Phi\hbf + \nbf
\end{equation}

Em que $\rbf_p=\Vect{R}$ é o vetor de símbolos recebidos, $\hbf=\Vect{H}$ é o vetor de canal e $\nbf$ é o ruído de transmissão. Para os símbolos, qualquer configuração ortogonal é válida. O artigo escolhe gerar a matriz a partir de uma matriz FFT.  

Para esse trabalho, são observados os desempenhos de três estimadores: Estimador de mínimos quadrados, Estimador AQN e o estimador proposto por~\cite{li.etal_2017a}. 

O estimador de mínimos quadrados é simplesmente o resultado de mínimos quadrados do sistema~\eqref{eq:min_squa}. Como essa estimação não considera a quantização dos símbolos recebidos, resultados de estimação desse estimador são calculados tanto para o caso dos símbolos recebidos terem sido quantizados ou não. O estimador AQN é uma consequência do estimador proposto para o caso específico da dimensão da matriz $\Phi$ ser quadrada, ou seja $\tau=K$, e sua equação é vista em~\cite[eq. 16]{li.etal_2017a}.

\subsubsection{Estimador de Bussgang}

Para o estimador proposto, é calculada a seguinte equação:

\begin{equation}
    \hat{h} = \tilde{\Phi}^H\Cbf_{\rbf_p}^{-1}\rbf_p       
    \label{eq:estimador_proposto}
\end{equation}

Em que $\Cbf_{\rbf_p}$ é a matriz de correlação da entrada quantizada, e $\rbf_p$ são os símbolos recebidos no processo de transmissão dos pilotos.

\begin{equation}
    \tilde{\Phi}=A_p\bar{\Phi}=A_p(\Phi\otimes\rho_pI_M)
\end{equation}

Onde $I_M$ é uma matriz identidade $M\times M$, $\rho_p$ é a potência de transmissão, $\otimes$ é o produto de Kronecker, e $A_p$ é a matriz de decomposição Bussgang, como descrita no relatório teórico.

Para calcular a matriz de correlação de $\rbf_p$, é necessário utilizar a relação da equação~\cite[eq. (13)]{li.etal_2017a} derivada da decomposição de bussgang para quantizadores de 1 bit. Essa equação depende da correlação do sinal antes da quantização $\ybf_p=\Vect{Y_p}$, assim, essa correlação é calculada diretamente como descrito abaixo:

\begin{align}
    \label{eq:calculo_correlacao_yp}
    \Cbf_{\ybf_p}   &= E[y_py_p^H] \\ \nonumber
                    &= E[(\Phi \hbf + \nbf)(\Phi \hbf + \nbf)^H] \\ \nonumber 
                    &= \Phi E[\hbf \hbf^H] \Phi^H + 2\Phi E[\hbf\nbf^H] + E[\nbf\nbf^H] \\ \nonumber 
                    &= \Phi \Cbf_\hbf \Phi^H + 2\Phi\Cbf_{\hbf\nbf} + \Cbf_{\nbf} \\ \nonumber 
                    &= \Phi \Cbf_\hbf \Phi^H + \Cbf_{\nbf}
\end{align}

Como tanto o modelo de canal quanto o ruído não assume correlação espacial entre os canais, temos que $\Cbf_{\hbf}=I_{M\tau}$. E como não há correlação entre o canal e o ruído, $\Cbf_{\hbf\nbf}=0$.

\subsection{Descrição dos experimentos}

O experimento para esitmação de canal é descrito pelo algoritmo~\ref{alg:estimacao_canal}. O experimento faz uma simulação de Monte Carlo de $Repetições$ repetições que gera um canal e utiliza todosos estimadores, para cada valor de SNR, para estimar o canal e medir o erro médio quadrático normalizado para cada estimação.  

No experimento, é utilizado a função \texttt{CalcularEstimador} que que representa a operação de realizar uma transmissão de $\tau$ pilotos através do canal real $H$ com um valor específico

\begin{algorithm}
    \label{alg:estimacao_canal}
    \KwData{$ListaSNR$, $ListaEstimadores$, $Repetições$, $m$, $k$, $\tau$}
    \KwResult{$Resultados$}
    $\Phi\gets$ \texttt{np.fft.fft}($\tau$,$k$)\Comment*[r]{Calcula a matriz de pilotos}
    \For{$i=0;\; i < Repetições;\; i=i+1$}{
        $H\gets$\texttt{GerarCanal($m$,$k$)}\;
        \For{$SNR$ em $ListaSNR$}{
            \For{$estimador$ em $ListaEstimadores$}{
                $Y_p\gets$ \texttt{TransmitirMIMO}($H$,$\Phi$)\;
                $Quantizar \gets$ \texttt{QuantizarTransmissao}($estimador$)\;
                \IfElse{$Quantizar$}{
                    $R_p \gets$ \texttt{Quantizar}($Y_p$)\;
                }{
                    $R_p \gets Y_p$\;
                }
                $\hat{H}\gets$ \texttt{CalcularEstimador($H$,$\Phi$,$estimador$)}\;
                $NMSE\gets (\texttt{SomarElementos(}||H-\hat{H}||_2^2\texttt{)})/mk$\;
                $NMSE_{dB}\gets 10*log(NMSE)$\;
                $resultados$\texttt{.concatenar(}$[i,SNR,estimador,NMSE_{dB}]]$\texttt{)}\;
            }
        }
    }
\end{algorithm}

\section{Capacidade do canal}
\label{sec:capacidade_canal}



\section{Considerações finais}
\label{sec:consideracoes_finais}



\printbibliography

\end{document}
