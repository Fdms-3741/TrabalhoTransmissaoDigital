\documentclass{article}
\usepackage{graphicx} % Required for inserting images

\usepackage[T1]{fontenc}
\usepackage[utf8]{inputenc} 
\usepackage[brazilian]{babel}

\usepackage[a4paper,total={6in,8in}]{geometry}

\usepackage{amsmath}
\usepackage{amssymb}

\usepackage[linesnumbered,ruled,portuguese]{algorithm2e}
\SetKwComment{Comment}{/* }{ */}
\SetEndCharOfAlgoLine{}
\SetKwInput{KwData}{Recebe}
\SetKwInput{KwResult}{Retorna}


% Usa as macros para vetores e matrizes
\usepackage{MathMacros}

\title{Relatório Final de Transmissão Digital}
\author{Fernando Dias de Mello Silva}
\date{6 de Dezembro 2023}

\begin{document}

\maketitle

\section{Introdução}

Esse relatório consiste na descrição das simulações e resultados obtidos do trabalho de~\cite{li.etal_2017a}. O trabalho consiste em avaliar o desempenho de estimadores de canal para o caso de sistemas \textit{massive} MIMO que fazem uso de quantizadores de 1 bit de resolução. A estimação de canal é feita com o uso da decomposição de Bussgang para obter um modelo linear estatísticamente equivalente do processo não linear de quantização. 

Esse relatório está dividio da seguinte forma:
Na seção~\ref{sec:ambiente_trabalho}, será apresentado o ambiente de simulação utilizado e as principais ferramentas, assim como formas de executar o código da simulação.
Na seção~\ref{sec:fundamentos_mimo}, será demonstrado o processo de simulação de um sistema MIMO convencional e serão demonstrados algumas das características desse sistema com o uso da ferramenta para validar o ambiente.
Na seção~\ref{sec:estimacao_canal}, será demonstrado quais as técnicas de estimação de canal utilizadas e qual o experimento feito para comparar as diferentes técnicas.
Na seção~\ref{sec:capacidade_canal}, será apresentado os modelos de otimização que serão inseridos em um solver e os resultados equivalentes.

Ao longo do trabalho, $\Cset$ é o conjunto dos números complexos e $\Rset$ o dos números reais. Vetores estão em negrito e são declarados como em $\xbf\in\Cset^{N}$. Matrizes, com $M$ linhas e $N$ colunas, estão em letras maiúsculas e são representadas como em $A\in\Rset^{M\times N}$. A função $\Vect{A}$ é a função de vetorização da matriz $A$. O operador $\otimes$ é o produto de Kroneker. A função $\Diag{\abf}$ é a função de diagonalização de um vetor $\abf$.  Quando um vetor ou uma matriz é aleatória, a notação $\sim$ indica a sua distribuição respectiva. $\Ncal(\mu,\sigma)$ é a distribuição normal no conjunto dos reais com média $\mu$ e desvio padrão $\sigma$, e a distribuição $\Ccal\Ncal(\mu,\sigma)$ é a distribuição normal complexa. 


\section{Ambiente de trabalho}
\label{sec:ambiente_trabalho}

A simulação foi feita com o uso da linguagem de programação \texttt{Python 3}. As principais bibliotecas utilizadas são as bibliotecas \texttt{NumPy}, \texttt{CuPy} e \texttt{SciPy} para cálculos numéricos e funções executadas em GPU, a biblioteca \texttt{Pandas} para manipulação de tabelas de resultados, a biblioteca \texttt{cvxpy} para otimização convexa e as bibliotecas \texttt{Matplotlib} e \texttt{Seaborn} para visualização dos resultados. Todas as funções das bibliotecas, quando citadas, serão atribuídas com o nome da biblioteca seguido de um ponto antes do nome da função (por exemplo: \texttt{biblioteca.função}). Quando não houver citação da biblioteca utilizada em uma função, a função foi desenvolvida por mim.

O ambiente de trabalho utilizado foi o \texttt{Jupyter}, esse ambiente permite a programação de ``\textit{notebooks}'' formados por uma estrutura de texto HTML e código que pode ser executado e interagido diretamente. A google oferece uma plataforma chamada Collab para a execução desses \textit{notebooks} em núvem. O código respectivo desse relatório está em um notebook do Collab que pode ser acessado através do link do repositório desse trabalho.

Os códigos, relatórios e outros materiais relacionados estão na minha página do github\footnote{https://github.com/Fdms-3741/TrabalhoTransmissaoDigital}.

\section{Fundamentos de MIMO}
\label{sec:fundamentos_mimo}

Nesta seção são apresentados as funções desenvolvidas para criação do ambiente de simulação de transmissão e estimação de canal de um sistema MIMO. 

\subsection{Definição das operações}

Como visto no relatório anterior, temos que a transmissão MIMO de $\tau$ símbolos entre $K$ transmissores e $M$ receptores consiste na seguinte equação:
\begin{equation}
    Y = \sqrt\rho HS + W
    \label{eq:transmissao_mimo}
\end{equation}

Em que $Y\in\Cset^{M\times\tau}$ são os símbolos recebidos pelos $M$ receptores, $S\in\Cset^{K\times\tau}$ são os símbolos transmitidos pelos $K$ transmissores, $W\in\Cset^{M\times\tau}\sim\Ccal\Ncal(0,1)$ é o ruído AWGN e $H\in\Cset^{M\times K}\sim\Ccal\Ncal(0,1)$ é a matriz de canal. 

A função \texttt{GerarBits(K,$\tau$)} que gera uma matriz $B$ de bits $2\tau\times K$ de distribuição uniforme. As funções \texttt{CodificarQPSK(MatrizBits)} e \texttt{DecodificarQPSK(MatrizSimbolos)} é responsável pela conversão entre matrizes de bits e matrizes de símbolos QPSK normalizados. A modulação consiste no mapeamento de dois bits ($\{11,10,01,00\}$) para um símbolo QPSK ($\{1+j,1-j,-1+j,-1-j\}1/\sqrt{2}$). Já a demodulação consiste na aplicação de um critério de decisão para o mapeamento de um número complexo $s$ qualquer para um número $s_{QPSK}$ contido na constelação antes de aplicar a operação inversa da modulação. Esse critério de decisão é definido como $s_{QPSK}=sign(Re(s))+sign(Im(s))$, onde $sign()$ é a função sinal.

Para realizar a transmissão MIMO, foi desenvolvida a função \texttt{TransmitirMIMO}, representada pelo algoritmo~\ref{alg:transmissao_mimo}. Ela recebe o canal, a matriz de sinal e a relação sinal ruído da transmissão. O ruído é gerado pela função descrita na linha~\ref{line:ruido} que gera uma matriz de dimensões $A\times B$ em que cada elemento é uma amostra de uma distribuição normal com média $\mu$ e desvio padrão $\sigma$. 

Nesse algoritmo, é possível ver que o ruído é normalizado ($\sigma=1$) para que a relação sinal ruído seja dada pelo ganho de transmissão $\rho$, que é calculado em função do SNR. O valor de $\rho$ também é dividido entre as antenas transmissoras, assim a potência de transmissão é dividida entre todas as antenas e a relação SNR se torna global. 

\begin{algorithm}
    \label{alg:transmissao_mimo}
    \caption{Função \texttt{TransmitirMIMO}}
    \KwData{$H,\sbf,SNR$}
    \KwResult{$\ybf$}
    \Comment{Exemplo de comentário. Variáveis e operadores serão representados por notação matemática convencional.}
    $\wbf \gets $\texttt{numpy.random.uniform(0,1,size=(M,$\tau$))}\; \label{line:ruido}
    $\rho \gets (10^{\frac{SNR}{10}})/K$\;
    $\ybf \gets \sqrt{\rho}H\sbf+\wbf$\;
\end{algorithm}

Já a recepção consiste em calcular o receptor $W^H$ necessário para obter o símbolo estimado $\hat{S}$ com base no sinal recebido pelas antenas $Y$. Esse processo é definido pela seguinte equação:

\begin{equation}
    S = W^TY
\end{equation}

Os dois receptores utilizados no trabalho são o \textit{Zero Forcing} (ZF), definido pela equação $H^{H}$, e o \textit{Maximum Ratio Combining} (MRC), definido pela equação $(H^HH)^{-1}H^T$. Cada um deles é calculado pelas funções \texttt{CalcularReceptor<Nome>($H$)}, em que \texttt{<Nome>} define qual receptor é utilizado (``ZF'' ou ``MRC'') e $H$ é a matriz de canal. Para os processos de estimação, a matriz de canal estimada $\hat{H}$ é utilizada no lugar da matriz de canal real.

A quantização é uma operação não linear aplicada pela função \texttt{Quantizar} assim como está definido em~\cite[eq. 2]{li.etal_2017a} e é aplicada na saída da operação de transmissão MIMO. Essencialmente, essa função apenas aplica o critério de decisão da função \texttt{DecodificarQPSK} mas sem converter os símbolos para bits. 

\subsection{BER do MIMO Convencional}

Para fazer uma prova de conceito das funções acima e investigar algumas características da divesidade espacial, fez-se dois experimentos para avaliar o BER de uma transmissão MIMO convencional em função do SNR e da proporção de antenas receptoras. 

O primeiro experimento é descrito pelo algoritmo~\ref{alg:ber_snr}. O experimento consiste em fazer uma simulação de Monte Carlo de todo o processo de transmissão MIMO. Cada experimetno consiste em gerar um novo canal $H$ e as matrizes de recepção para o ZF ($W^T_{ZF}$) e o MRC ($W^T_{MRC}$). Após isso, para cada valor de SNR, é gerado uma matriz de bits $B$ que é codificada, transmitida, recebida (para cada receptor) e decodificada. Esse processo resulta na matriz de bits estimada $\hat{B}$, que pode ser comparada com a matriz $B$ para o cálculo do BER dessa transmissão. As variáveis são definidas 

\begin{algorithm}
    \label{alg:ber_snr}
    \caption{Estimativa do BER via MonteCarlo}
    \KwData{$m,k,\tau,\text{repetições},\text{ListaM},\text{ListaSNR}$}
    \KwResult{$resultados$}
    $resultados = []$\Comment*[l]{Lista vazia para receber resultados}
    \Comment{Início das simulações}
    \For{(i=0;$i<\text{repetições};i=i+1$)}{ \label{line:ber_snr_rep_sim}
        \For{$M$ \textbf{em} ListaM}{
            \Comment{Canal novo gerado por iteração}
            $H \gets$ \texttt{GerarCanal($m$,$k$)}\;
            \Comment{Receptores calculados a partir do canal real}
            $W^T_{MRC} \gets$ \texttt{CalcularReceptorMRC(H)}\;
            $W^T_{ZF} \gets$ \texttt{CalcularReceptorMRC(H)}\;
            \For{$SNR$ \textbf{em} ListaSNR}{
                \Comment{Bits novos gerado por transmissão}
                $B \gets$ \texttt{GerarBits($k$,$\tau$)}\;
                $S \gets$ \texttt{CodificarQPSK($B$)}\;
                $Y \gets$ \texttt{TransmissaoMIMO(H,S,SNR)}\;
                \For{$receptor$ \textbf{em} $Receptores$}{
                    $\hat{S} \gets W^T_{receptor}Y$\;
                    $\hat{B} \gets$ \texttt{DecodificarQPSK($S$)}\;
                    $BER \gets$ \texttt{SomarElementos($|\hat{B}-B|$)}$/(k.\tau)$\;
                    \Comment{Adiciona o resultado dessa iteração na lista de resultados}
                    $resultados$\texttt{.concatenar($[i,SNR,receptor,BER]$)}\;
                }
            }
        }
    }
\end{algorithm}

Já o segundo experimento consiste em 

\section{Estimação de canal}
\label{sec:estimacao_canal}

A estimação de canal é feita com a transmissão de símbolos piloto, representados pela matriz $\Phi\in\Cset^{\tau\times K}$, e a equação de transmissão é a mesma vista na eq.~\eqref{eq:transmissao_mimo}. 

Para esse trabalho, são observados os desempenhos de três estimadores: Estimador de mínimos quadrados, Estimador AQN e o estimador proposto por~\cite{li.etal_2017a}. 

O estimador de mínimos quadrados simplesmente resolve o sistema de equações definido por~\cite[eq. 4]{li.etal_2017a}. Ele é definido pela função \texttt{EstimarCanalMS} e está definido no algoritmo~\ref{alg:estimador_ms}. Já que esse estimador serve apenas para o caso de MIMO convencional, a função \texttt{EstimarCanalMS} possui um parâmetro booleano chamado $Quantizar$ que define se o resultado da transmissão é quantizado antes de ser utilizado na equação de mínimos quadrados. 

O estimador AQN é uma consequência do estimador proposto para o caso específico de $\tau=K$, como é descrito na equação~\cite[eq]{li.ets} 
, e a equação de transmissão é a mesma vista na eq.~\eqref{eq:transmissao_mimo}. 

Para esse trabalho, são observados os desempenhos de três estimadores: Estimador de mínimos quadrados, Estimador AQN e o estimador proposto por~\cite{li.etal_2017a}. 

O estimador de mínimos quadrados simplesmente resolve o sistema de equações definido por~\cite[eq. 4]{li.etal_2017a}. Ele é definido pela função \texttt{EstimarCanalMS} e está definido no algoritmo~\ref{alg:estimador_ms}. Já que esse estimador serve apenas para o caso de MIMO convencional, a função \texttt{EstimarCanalMS} possui um parâmetro booleano chamado $Quantizar$ que define se o resultado da transmissão é quantizado antes de ser utilizado na equação de mínimos quadrados. 

O estimador AQN é uma consequência do estimador proposto para o caso específico de $\tau=K$, como é descrito na equação~\cite[eq]{li.ets} 
, e a equação de transmissão é a mesma vista na eq.~\eqref{eq:transmissao_mimo}. 

Para esse trabalho, são observados os desempenhos de três estimadores: Estimador de mínimos quadrados, Estimador AQN e o estimador proposto por~\cite{li.etal_2017a}. 

O estimador de mínimos quadrados simplesmente resolve o sistema de equações definido por~\cite[eq. 4]{li.etal_2017a}. Ele é definido pela função \texttt{EstimarCanalMS} e está definido no algoritmo~\ref{alg:estimador_ms}. Já que esse estimador serve apenas para o caso de MIMO convencional, a função \texttt{EstimarCanalMS} possui um parâmetro booleano chamado $Quantizar$ que define se o resultado da transmissão é quantizado antes de ser utilizado na equação de mínimos quadrados. 

O estimador AQN é uma consequência do estimador proposto para o caso específico de $\tau=K$, como é descrito na equação~\cite[eq]{li.ets} 
, e a equação de transmissão é a mesma vista na eq.~\eqref{eq:transmissao_mimo}. 

Para esse trabalho, são observados os desempenhos de três estimadores: Estimador de mínimos quadrados, Estimador AQN e o estimador proposto por~\cite{li.etal_2017a}. 

O estimador de mínimos quadrados simplesmente resolve o sistema de equações definido por~\cite[eq. 4]{li.etal_2017a}. Ele é definido pela função \texttt{EstimarCanalMS} e está definido no algoritmo~\ref{alg:estimador_ms}. Já que esse estimador serve apenas para o caso de MIMO convencional, a função \texttt{EstimarCanalMS} possui um parâmetro booleano chamado $Quantizar$ que define se o resultado da transmissão é quantizado antes de ser utilizado na equação de mínimos quadrados. 

O estimador AQN é uma consequência do estimador proposto para o caso específico de $\tau=K$, como é descrito na equação~\cite[eq]{li.ets} 

\section{Capacidade do canal}
\label{sec:capacidade_canal}

\bibliographystyle{ieee} % We choose the "plain" reference style
\bibliography{Transmissão Digital}

\end{document}
