\documentclass{article}
\usepackage{graphicx} % Required for inserting images

\usepackage[T1]{fontenc}
\usepackage[utf8]{inputenc} 
\usepackage[brazilian]{babel}

\usepackage{amsmath}
\usepackage{amssymb}

\usepackage[ruled,portuguese]{algorithm2e}
\SetKwComment{Comment}{/* }{ */}
\SetKwInput{KwData}{Recebe}
\SetKwInput{KwResult}{Retorna}


% Usa as macros para vetores e matrizes
\usepackage{MathMacros}

\title{Relatório Final de Transmissão Digital}
\author{Fernando Dias de Mello Silva}
\date{6 de Dezembro 2023}

\begin{document}

\maketitle

\section{Introdução}

Esse relatório consiste na descrição das simulações e resultados obtidos do trabalho de~\cite{li.etal_2017a}. O trabalho consiste em avaliar o desempenho de estimadores de canal para o caso de sistemas \textit{massive} MIMO que fazem uso de quantizadores de 1 bit de resolução. A estimação de canal é feita com o uso da decomposição de Bussgang para obter um modelo linear estatísticamente equivalente do processo não linear de quantização. 

As simulações foram feitas com o uso da linguagem \texttt{Python 3}, com o uso das bibliotecas \texttt{Numpy} para cálculos matriciais, \texttt{}

Esse relatório está dividio da seguinte forma:
Na seção~\ref{sec:ambiente_trabalho}, será apresentado o ambiente de simulação utilizado e as principais ferramentas, assim como formas de executar o código da simulação.
Na seção~\ref{sec:fundamentos_mimo}, será demonstrado o processo de simulação de um sistema MIMO convencional e serão demonstrados algumas das características desse sistema com o uso da ferramenta para validar o ambiente.
Na seção~\ref{sec:estimacao_canal}, será demonstrado quais as técnicas de estimação de canal utilizadas e qual o experimento feito para comparar as diferentes técnicas.
Na seção~\ref{sec:capacidade_canal}, será apresentado os modelos de otimização que serão inseridos em um solver e os resultados equivalentes.

Ao longo do trabalho, $\Cset$ é o conjunto dos números complexos e $\Rset$ o dos números reais. Vetores estão em negrito e são declarados como em $\xbf\in\Cset^{N}$. Matrizes, com $M$ linhas e $N$ colunas, estão em letras maiúsculas e são representadas como em $A\in\Rset^{M\times N}$. A função $\Vect{A}$ é a função de vetorização da matriz $A$. O operador $\otimes$ é o produto de Kroneker. A função $\Diag{\abf}$ é a função de diagonalização de um vetor $\abf$.  Quando um vetor ou uma matriz é aleatória, a notação $\sim$ indica a sua distribuição respectiva. $\Ncal(\mu,\sigma)$ é a distribuição normal no conjunto dos reais com média $\mu$ e desvio padrão $\sigma$, e a distribuição $\Ccal\Ncal(\mu,\sigma)$ é a distribuição normal complexa. 


\section{Ambiente de trabalho}
\label{sec:ambiente_trabalho}

A simulação foi feita com o uso da linguagem de programação \texttt{Python 3}. As principais bibliotecas utilizadas são as bibliotecas \texttt{NumPy}, \texttt{CuPy} e \texttt{SciPy} para cálculos numéricos e funções executadas em GPU, a biblioteca \texttt{Pandas} para manipulação de tabelas de resultados, a biblioteca \texttt{cvxpy} para otimização convexa e as bibliotecas \texttt{Matplotlib} e \texttt{Seaborn} para visualização dos resultados. Todas as funções das bibliotecas, quando citadas, serão atribuídas com o nome da biblioteca seguido de um ponto antes do nome da função (por exemplo: \texttt{biblioteca.função}). Quando não houver citação da biblioteca utilizada em uma função, a função foi desenvolvida por mim.

O ambiente de trabalho utilizado foi o \texttt{Jupyter}, esse ambiente permite a programação de ``\textit{notebooks}'' formados por uma estrutura de texto HTML e código que pode ser executado e interagido diretamente. A google oferece uma plataforma chamada Collab para a execução desses \textit{notebooks} em núvem. O código respectivo desse relatório está em um notebook do Collab\footnote{}.

Os códigos, relatórios e outros materiais relacionados estão na minha página do github\footnote{https://github.com/Fdms-3741/TrabalhoTransmissaoDigital}.

\section{Fundamentos de MIMO}
\label{sec:fundamentos_mimo}

Essa seção faz a implementação de casos mínimos e testa na sua testa valores diferentes de testa bem testada.

\subsection{Definição das operações}

Como visto no relatório anterior, temos que a transmissão MIMO de $\tau$ símbolos entre $K$ transmissores e $M$ receptores consiste na seguinte equação:
\begin{equation}
    Y = \sqrt\rho HS + W
    \label{eq:transmissao_mimo}
\end{equation}

Em que $Y\in\Cset^{M\times\tau}$ são os símbolos recebidos pelos $M$ receptores, $S\in\Cset^{K\times\tau}$ são os símbolos transmitidos pelos $K$ transmissores, $W\in\Cset^{M\times\tau}\sim\Ccal\Ncal(0,1)$ é o ruído AWGN e $H\in\Cset^{M\times K}\sim\Ccal\Ncal(0,1)$ é a matriz de canal. 

Os símbolos são gerados a partir de uma função \texttt{GerarBits(M,N)} que gera uma matriz $M\times N$ de bits $B$ e uma função \texttt{ModularQPSK(MatrizBits)} que recebe essa matriz de bits e retorna uma matriz $S$ com os símbolos correspondentes. Uma transmissão MIMO é feita pela função \texttt{TransmitirMIMO}, representada pelo algoritmo~\ref{alg:transmissao_mimo}.

\begin{algorithm}
    \label{alg:transmissao_mimo}
    \caption{Função \texttt{TransmitirMIMO}}
    \KwData{$H,\sbf,SNR$}
    \KwResult{$\ybf$}
    \Comment{Exemplo de comentário. Variáveis e operadores serão representados por notação matemática convencional.}
    $\wbf \gets $GerarRuído(0,1)\;
    $\rho \gets (10^{\frac{SNR}{10}})/K$\;
    $\ybf \gets \sqrt{\rho}H\sbf+\wbf$\;
\end{algorithm}

Já a recepção para calcular o símbolo estimado $\hat{S}$ consiste em calcular o receptor $W^H$ com base na equação $\hat{S}=W^HY$ deve ser feita com um dos receptores propostos: o \textit{Zero Forcing} (ZF) pela equação $H^{H}$ e o \textit{Maximum Ratio Combining} (MRC) pela equação $(H^HH)^{-1}H^T$. Cada um deles é calculado pelas funções \texttt{CalcularReceptor(Nome,$H$)}, em que \texttt{Nome} define qual receptor é utilizado(``ZF'' ou ``MRC'') e $H$ é a matriz de canal, que pode ser tanto a real quanto a estimada.

A quantização é uma operação não linear aplicada pela função \texttt{Quantizar} assim como está definido em~\cite e é aplicada na saída da operação de transmissão MIMO. Assim, 

\subsection{BER do MIMO Convencional}

Para fazer uma prova de conceito das funções acima e investigar algumas características da divesidade espacial, fez-se dois experimentos para avaliar o BER de uma transmissão MIMO convencional em função do SNR e da proporção de antenas receptoras. 

O primeiro experimento consiste 

\section{Estimação de canal}
\label{sec:estimacao_canal}

A estimação de canal é feita com a transmissão de símbolos piloto, representados pela matriz $\Phi\in\Cset^{\tau\times K}$, e a equação de transmissão é a mesma vista na eq.~\eqref{eq:transmissao_mimo}. 

Para esse trabalho, são observados os desempenhos de três estimadores: Estimador de mínimos quadrados, Estimador AQN e o estimador proposto por~\cite{li.etal_2017a}. 

O estimador de mínimos quadrados simplesmente resolve o sistema de equações definido por~\cite[eq. 4]{li.etal_2017a}. Ele é definido pela função \texttt{EstimarCanalMS} e está definido no algoritmo~\ref{alg:estimador_ms}. Já que esse estimador serve apenas para o caso de MIMO convencional, a função \texttt{EstimarCanalMS} possui um parâmetro booleano chamado $Quantizar$ que define se o resultado da transmissão é quantizado antes de ser utilizado na equação de mínimos quadrados. 

O estimador AQN é uma consequência do estimador proposto para o caso específico de $\tau=K$, como é descrito na equação~\cite[eq]{li.ets} 
, e a equação de transmissão é a mesma vista na eq.~\eqref{eq:transmissao_mimo}. 

Para esse trabalho, são observados os desempenhos de três estimadores: Estimador de mínimos quadrados, Estimador AQN e o estimador proposto por~\cite{li.etal_2017a}. 

O estimador de mínimos quadrados simplesmente resolve o sistema de equações definido por~\cite[eq. 4]{li.etal_2017a}. Ele é definido pela função \texttt{EstimarCanalMS} e está definido no algoritmo~\ref{alg:estimador_ms}. Já que esse estimador serve apenas para o caso de MIMO convencional, a função \texttt{EstimarCanalMS} possui um parâmetro booleano chamado $Quantizar$ que define se o resultado da transmissão é quantizado antes de ser utilizado na equação de mínimos quadrados. 

O estimador AQN é uma consequência do estimador proposto para o caso específico de $\tau=K$, como é descrito na equação~\cite[eq]{li.ets} 
, e a equação de transmissão é a mesma vista na eq.~\eqref{eq:transmissao_mimo}. 

Para esse trabalho, são observados os desempenhos de três estimadores: Estimador de mínimos quadrados, Estimador AQN e o estimador proposto por~\cite{li.etal_2017a}. 

O estimador de mínimos quadrados simplesmente resolve o sistema de equações definido por~\cite[eq. 4]{li.etal_2017a}. Ele é definido pela função \texttt{EstimarCanalMS} e está definido no algoritmo~\ref{alg:estimador_ms}. Já que esse estimador serve apenas para o caso de MIMO convencional, a função \texttt{EstimarCanalMS} possui um parâmetro booleano chamado $Quantizar$ que define se o resultado da transmissão é quantizado antes de ser utilizado na equação de mínimos quadrados. 

O estimador AQN é uma consequência do estimador proposto para o caso específico de $\tau=K$, como é descrito na equação~\cite[eq]{li.ets} 
, e a equação de transmissão é a mesma vista na eq.~\eqref{eq:transmissao_mimo}. 

Para esse trabalho, são observados os desempenhos de três estimadores: Estimador de mínimos quadrados, Estimador AQN e o estimador proposto por~\cite{li.etal_2017a}. 

O estimador de mínimos quadrados simplesmente resolve o sistema de equações definido por~\cite[eq. 4]{li.etal_2017a}. Ele é definido pela função \texttt{EstimarCanalMS} e está definido no algoritmo~\ref{alg:estimador_ms}. Já que esse estimador serve apenas para o caso de MIMO convencional, a função \texttt{EstimarCanalMS} possui um parâmetro booleano chamado $Quantizar$ que define se o resultado da transmissão é quantizado antes de ser utilizado na equação de mínimos quadrados. 

O estimador AQN é uma consequência do estimador proposto para o caso específico de $\tau=K$, como é descrito na equação~\cite[eq]{li.ets} 

\section{Capacidade do canal}
\label{sec:capacidade_canal}



\end{document}
