\documentclass{article}

\usepackage{amsmath}
\usepackage{amsfonts}

\usepackage{MathMacros}


\title{Análise de quantizadores de baixa resolução em sistemas Massive MIMO}
\author{Fernando Dias}
\date{}

\begin{document}
	\maketitle
	
	\section{Introdução}
	Em sistemas de transmissão com apenas uma antena transmissora e uma antena receptora, o sistema está sujeito as condições do canal em um determinado instante, o que em casos de \textit{deep fade} pode reduzir consideravelmente o desempenho. As dimensões exploradas para contornar esse problema até então são no tempo, com codificações do símbolo, e na frequência, com ODFM. Sistemas MIMO introduzem uma nova dimensão, a espacial, no processo de comunicação. Múltiplas antenas suficientemente espaçadas tem como objetivo fazer com que a comunicação aconteça simultâneamente em múltiplos canais que são independentes entre si, o que diminui a chance de erro de detecção. Uma outra vantagem de se ter múltiplas antenas é o \textit{beamforming}, que é o resultado de um atraso do sinal entre as antenas que permite que o sinal seja direcionado para uma região específica ao invés de ser omnidirecional.

	Para tecnologias como 5G e possivelmente o 6G, explora-se o conceito de \textit{massive} MIMO, que consiste em um uso massivo de antenas na estação base. Esse conceito visa explorar comunicações em mais altas frequências (tais como a \textit{mmWave}) que, por consequência, reduzem o tamanho da antena e com isso permitem agregar dezenas ou centenas de antenas em um espaço viável. Esse processo tem inúmeras vantagens tais como o \textit{beamforming} bem estreito e a redução de potência por antena. Porém, a densidade de antenas promove desafios nos conversores analógico digital (ADC), que por ser replicada e condensada nesssa escala ela desperdiça muita energia. 

	Um possível problema nesse tipo de sistema é a necessidade de utilizar ADCs de baixa resolução para reduzir a complexidade, custo e gasto energético na cadeia de rádio-frequência. O caso particular de conversores de um bit pode ser implementado apenas com um comparador e assim permite que a rede não precise realizar controle de ganho. A consequência desse ADC é a presença de um componente não linear na resposta da recepção, que impactará tanto na estimação quanto na capacidade do canal.

	Com isso, o trabalho de~\cite{li.etal_2017a} apresenta um método de estimação de canal que considera os efeitos não lineares dos ADCs na estimativa. O trabalho propõe uma aproximação linear do processo de quantização e um estimador de canal que compensa o efeito das não-linearidades. Esse estimador é comparado com outros estimadores da literatura em relação a capacidade do canal e a eficiência espectral

	Esse texto está organizado da seguinte forma:
	Na seção~\ref{sec:modelo_sistema}, será discutido quais são as formas 

	\section{Modelo do sistema}
	\label{sec:modelo_sistema}

	Com isso, o sinal recebido pelo receptor possui a forma:
	\begin{equation}
		\vec{y}=\sqrt{\rho_t}H\vec{x}+\vec{\omega}
		\label{eq:recepcao_sinal_mimo}
	\end{equation}
	Dado que o receptor tem $M$ antenas e o transmissor tem $N$ antenas, a matriz de canal complexa é $H\in\mathbb{C}^{M\times N}$
	

	\section{Métricas de avaliação}
	\label{sec:metricas_avaliacao}
		

	\section{Proposta de trabalho}
	\label{sec:proposta_trabalho}


	\bibliographystyle{ieeetr}
	\bibliography{Trabalho}

\end{document}
